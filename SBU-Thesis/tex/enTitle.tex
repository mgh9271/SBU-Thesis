% !TeX root=../main.tex
% در این فایل، عنوان پایان‌نامه، مشخصات خود و چکیده پایان‌نامه را به انگلیسی، وارد کنید.

%%%%%%%%%%%%%%%%%%%%%%%%%%%%%%%%%%%%
\latinuniversity{Shahid Beheshti University}
\latincollege{College of Engineering}
\latinfaculty{Faculty of Electrical Engineering}
\latindepartment{Algorithms and Computation}
\latinsubject{Electrical Engineering}
\latinfield{Electric Transportation Systems}
\latintitle{Optimal Coordination between an Electric-Vehicle Charging Station and a Battery Swap Station}
\firstlatinsupervisor{Dr Morteza Kheradmandi}
%\secondlatinsupervisor{Second Supervisor}
%\firstlatinadvisor{Dr Maziar karimi}
%\secondlatinadvisor{Second Advisor}
\latinname{Mohammad Mahdi}
\latinsurname{Gholami}
\latinthesisdate{Feb 2020}
\latinkeywords{Fast Charging Station, Monte Carlo, Battery Swap Station, Planing, Electric Vehicle}
\en-abstract{\justifying
Transportation 	sector is the largest consumer of petroleum and other liquids, particularly motor
gasoline and distillate fuel oil. Motor gasoline and distillate fuel oil hold a combined share of 84 percent of the total transportation energy consumption. Environmental concerns have have shifted focus to less polluting transportation methods.	The Electric Vehicles, in addition to superior technical performance, have produced promising results in reducing carbon emissions.
 Widespread adoption of electric vehicles, on the other hand, significantly increases the load demand on the power grid, and thus wide deployment of electric vehicles requires comprehensive planning for charging infrastructures to minimize the damage caused by overall power outages or part of the load.  Therefore, accurate and realistic prediction of driver behavior and consequently the calculation of energy demand in the planning process of charging infrastructure will be necessary.
Two common and studied methods in electric car charging methods are fast charging and battery swap station. These studies are aimed at resolving the problem of charging speed and battery life . Due to the bankruptcy of one of the big and reputable battery replacement companies and on the other hand the expansion of fast charging infrastructure, it seems more reasonable for personal vehicle to use the fast charging infrastructure. But recent studies and operational plans in Beijing has showed battery swapping infrastructure potentials for public transportation.
The current study looks at the charging network in the presence of
 fast charging stations and battery swap stations, and examines drivers' behavior by presenting a probabilistic model. The two-stage probabilistic model simulates user behavior. In the first step, the Monte Carlo probabilistic method is used to model the driver charging decision and then the cost-benefit method is used to model charging station selection. Finally, the daily queue status of the charging stations and the economic factors for the stations are evaluated.}
