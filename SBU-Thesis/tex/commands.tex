% در این فایل، دستورها و تنظیمات مورد نیاز، آورده شده است.
%-------------------------------------------------------------------------------------------------------------------
% دستوراتی که پوشه پیش‌فرض زیرفایل‌های tex را مشخص می‌کند.
%\makeatletter
%\def\input@path{{./tex/}}
%\makeatother

% در ورژن جدید زی‌پرشین برای تایپ متن‌های ریاضی، این سه بسته، حتماً باید فراخوانی شود
\usepackage{amsthm,amssymb,amsmath}
\usepackage{mathtools}
\DeclarePairedDelimiter\set\{\}
% بسته‌ای برای تنطیم حاشیه‌های بالا، پایین، چپ و راست صفحه
\usepackage[top=40mm, bottom=40mm, left=25mm, right=35mm]{geometry}
% بسته‌‌ای برای ظاهر شدن شکل‌ها و تعیین آدرس تصاویر
\usepackage[final]{graphicx}
\usepackage{multirow}
\graphicspath{{./img/}}
% بسته‌های مورد نیاز برای نوشتن کدها، رنگ‌آمیزی آنها و تعیین پوشهٔ کدها
\usepackage[final]{listings}
\usepackage[usenames,dvipsnames,svgnames,table]{xcolor}
\lstset{inputpath=./code/}

% بسته‌ای برای رسم کادر
\usepackage{ragged2e}
\usepackage{url}
\usepackage{framed} 
% بسته‌‌ای برای چاپ شدن خودکار تعداد صفحات در صفحه «معرفی پایان‌نامه»
\usepackage{lastpage}
% بسته‌ٔ لازم برای: ۱. تغییر شماره‌گذاری صفحات پیوست. ۲. تصحیح باگ آدرس وب حاوی '%' در مراجع
\usepackage{etoolbox}
\usepackage{graphicx}
%%%%%%%%%%%%%%%%%%%%%%%%%%%%%%%%%%%%%%%%
\usepackage{amsmath}
%%%%%%%%%%%%%%%%%%%%%%%%%%%%%%%%%%%%%%%%
\usepackage{tikz}
\usetikzlibrary{chains,shapes.multipart}
\usetikzlibrary{shapes,calc}
\usetikzlibrary{automata,positioning}
\usepackage{subfig}
\usepackage{booktabs}
%\usepackage{subcaption}
\usepackage{blindtext}
\usepackage{graphicx}
\usepackage{tikz}
\usetikzlibrary{calc,shapes,arrows}
\usepackage{bm,times}
\newcommand{\mx}[1]{\mathbf{\bm{#1}}} % Matrix command
\newcommand{\vc}[1]{\mathbf{\bm{#1}}} % Vector command

\def\cents{\hbox{\rm\rlap/c}}
\usepackage{pgfplots}
%%%%%%%%%%%%%%%%%%%%%%%%%%%%%%%%%%%%
%%% دستورات وابسته به استیل مراجع:
\newcounter{plotno}
\usepackage{pgfplotstable}
\def\pgfplotsinvokeiflessthan#1#2#3#4{%
	\pgfkeysvalueof{/pgfplots/iflessthan/.@cmd}{#1}{#2}{#3}{#4}\pgfeov
}%
\def\pgfplotsmulticmpthree#1#2#3#4#5#6\do#7#8{%
	\pgfplotsset{float <}%
	\pgfplotsinvokeiflessthan{#1}{#4}{%
		% first key <:
		#7%
	}{%
		\pgfplotsinvokeiflessthan{#4}{#1}{%
			% first key >:
			#8%
		}{%
			% first key ==:
			\pgfplotsset{float <}%
			\pgfplotsinvokeiflessthan{#2}{#5}{%
				% second key <
				#7%
			}{%
				\pgfplotsinvokeiflessthan{#5}{#2}{%
					% second key >
					#8%
				}{%
					% second key ==
					\pgfplotsset{float <}%
					\pgfplotsinvokeiflessthan{#3}{#6}{%
						% third key <
						#7%
					}{%
						% third key >=
						#8%
					}%
				}%
			}%
		}%
	}%
}%




\usepackage{filecontents}
%%% اگر از استیل‌های natbib (plainnat-fa، asa-fa، chicago-fa) استفاده می‌کنید، خط زیر را فعال و بعدی‌اش را غیرفعال کنید.
%\usepackage{natbib}
%\newcommand{\citelatin}[1]{\cite{#1}\LTRfootnote{\citeauthor*{#1}}}
%\newcommand{\citeplatin}[1]{\citep{#1}\LTRfootnote{\citeauthor*{#1}}}
%%% اگر از سایر استیل‌ها استفاده می‌کنید، خط بالا را غیرفعال و خط‌های زیر را فعال کنید.

%ًُ\bibliographystyle{unsrt}
\let\citep\cite
\let\citelatin\cite
\let\citeplatin\cite
%%%%%%%%%%%%
\usepackage{titlesec}
\titlelabel{\thetitle.\quad}
% بررسی حالت پیش نویس
\usepackage{ifdraft}
\ifdraft
{%
	% بسته‌ٔ ایجاد لینک‌های رنگی با امکان جهش
	\usepackage{hyperref}
\hypersetup{hidelinks}

%\usepackage[unicode=true,pagebackref=true,
%colorlinks,linkcolor=blue,citecolor=blue,final]{hyperref}

	%\usepackage{todonotes}
	\usepackage[firstpage]{draftwatermark}
	\SetWatermarkText{\ \ \ پیش‌نویس}
	\SetWatermarkScale{1.2}
}
{ 
	\usepackage{hyperref}
	\hypersetup{hidelinks}
%\usepackage[pagebackref=false,colorlinks,
%linkcolor=blue,citecolor=blue,urlcolor=blue]{hyperref}

	%\usepackage[disable]{todonotes} % final without TODOs
}

\usepackage[obeyDraft]{todonotes}
\setlength{\marginparwidth}{2cm}
\def\infinity{\rotatebox{90}{8}}
%%%%%%%%%%%%
%%% تصحیح باگ: اگر در مراجع، آدرس وب حاوی '%' بوده و pagebackref فعال باشد، دستورات زیر باید بیایند:
%% برای استیل‌های natbib مثل plainnat-fa، asa-fa، chicago-fa
\makeatletter
\let\ORIG@BR@@lbibitem\BR@@lbibitem
\apptocmd\ORIG@BR@@lbibitem{\endgroup}{}{}
\def\BR@@lbibitem{\begingroup\catcode`\%=12 \ORIG@BR@@lbibitem}
\makeatother
%% برای سایر استیل‌ها
\makeatletter
\let\ORIG@BR@@bibitem\BR@@bibitem
\apptocmd\ORIG@BR@@bibitem{\endgroup}{}{}
\def\BR@@bibitem{\begingroup\catcode`\%=12 \ORIG@BR@@bibitem}
\makeatother
%%%%%%%%%%%%%%%%%%%%%%%%%%%%%%%%%%%%


% بسته‌ لازم برای تنظیم سربرگ‌ها
\usepackage{fancyhdr}
%\usepackage{enumitem}
\usepackage{setspace}
\usepackage[displaymath,textmath,sections,graphics,floats]{preview}
\PreviewEnvironment{enumerate}
% بسته‌های لازم برای نوشتن الگوریتم
\usepackage{algorithm}
\usepackage{algorithmic}
% بسته‌های لازم برای رسم بهتر جداول
\usepackage{tabulary}
\usepackage{tabularx}
\usepackage{rotating}
% بسته‌های لازم برای رسم تنظیم بهتر شکل‌ها و زیرشکل‌ها
\usepackage[export]{adjustbox}
\usepackage{pdflscape}
\usepackage{subfig}
\usepackage{lipsum,multicol}
\usepackage{supertabular}
\let\counterwithout\relax
\let\counterwithin\relax
\usepackage{chngcntr}  
\usepackage[subfigure]{tocloft}
\usepackage{ragged2e}



\counterwithout{figure}{chapter}
\counterwithout{table}{chapter}

\renewcommand{\cftfigpresnum}{شکل\ }
\renewcommand{\cfttabpresnum}{جدول\ }

\newlength{\mylenf}
\settowidth{\mylenf}{\cftfigpresnum}
\setlength{\cftfignumwidth}{\dimexpr\mylenf+3.5em}
\setlength{\cfttabnumwidth}{\dimexpr\mylenf+3.5em}


\makeatletter
\newcommand\listoftablesandfigures{%
	\chapter*{List of Tables and Figures}%
	\phantomsection
	\@starttoc{lof}%
	\bigskip
	\@starttoc{lot}}
\makeatother

%\usepackage{tocloft}
%\newlength{\mylen}
%\renewcommand{\cftfigpresnum}{\figurename\enspace}
%\renewcommand{\cftfigaftersnum}{:}
%\settowidth{\mylen}{\cftfigpresnum\cftfigaftersnum}
%\addtolength{\cftfignumwidth}{\mylen}
%

\usepackage[none]{hyphenat}
% بسته‌ای برای رسم نمودارها و نیز صفحه مالکیت اثر
\usepackage{tikz}
% بسته‌ای برای ظاهر شدن «مراجع» و «نمایه» در فهرست مطالب
\usepackage[nottoc]{tocbibind}
% دستورات مربوط به ایجاد نمایه
\usepackage{makeidx}
\makeindex
%%% بسته ایجاد واژه‌نامه با xindy
\usepackage[xindy,acronym,nonumberlist=true]{glossaries}
% بسته زیر باگ ناشی از فراخوانی بسته‌های زیاد را برطرف می‌کند.
\usepackage{morewrites}
%%%%%%%%%%%%%%%%%%%%%%%%%%
% فراخوانی بسته زی‌پرشین (باید آخرین بسته باشد)
\usepackage[extrafootnotefeatures, localise=on, displaymathdigits=persian]{xepersian}
\ExplSyntaxOn \cs_set_eq:NN \etex_iffontchar:D \tex_iffontchar:D \ExplSyntaxOff


%\usepackage[perpage]{footmisc}

\usepackage{perpage} %the perpage package
\MakePerPage{footnote} %the perpage package command

\usepackage{booktabs}% http://ctan.org/pkg/booktabs
\newcommand{\mytab}{% Just for this example
	\resizebox{0.46\textwidth}{!}{%
	\begin{tabular}{
		>{\columncolor[HTML]{C0C0C0}}c |
		>{\columncolor[HTML]{EFEFEF}}c |ccc|}
	\hline
	\multicolumn{2}{|c|}{\cellcolor[HTML]{C0C0C0}} & \multicolumn{3}{c|}{\cellcolor[HTML]{C0C0C0}نقطهٔ شارژی اول} \\ \cline{3-5} 
	\multicolumn{2}{|c|}{\multirow{-2}{*}{\cellcolor[HTML]{C0C0C0}زمان سرویس دهی در نقطهٔ شارژی اول}} & \cellcolor[HTML]{EFEFEF}1 & \cellcolor[HTML]{EFEFEF}2 & \cellcolor[HTML]{EFEFEF}3 \\ \hline
	\cellcolor[HTML]{C0C0C0} & 1 & 52٫0 & 18٫1 & 8٫9 \\
	\cellcolor[HTML]{C0C0C0} & 2 & 41٫7 & 14٫1 & 8٫1 \\
	\multirow{-3}{*}{\cellcolor[HTML]{C0C0C0}نقطهٔ شارژی دوم} 
	& 3 & 29٫7& 12٫4 & 7٫8 \\ \hline
\end{tabular}%
}
}
\newcommand{\mytaba}{% Just for this example
	\resizebox{0.46\textwidth}{!}{%
			\begin{tabular}{|
			>{\columncolor[HTML]{C0C0C0}}c |
			>{\columncolor[HTML]{EFEFEF}}c |ccc|}
		\hline
		\multicolumn{2}{|c|}{\cellcolor[HTML]{C0C0C0}} & \multicolumn{3}{c|}{\cellcolor[HTML]{C0C0C0}نقطهٔ شارژی اول } \\ \cline{3-5} 
		\multicolumn{2}{|c|}{\multirow{-2}{*}{\cellcolor[HTML]{C0C0C0}زمان سرویس‌دهی در نقطهٔ شارژی دوم}} & \cellcolor[HTML]{EFEFEF}1 & \cellcolor[HTML]{EFEFEF}2 & \cellcolor[HTML]{EFEFEF}3 \\ \hline
		\cellcolor[HTML]{C0C0C0} & 1 & 84٫7 & 55٫2 & 37٫7\\
		\cellcolor[HTML]{C0C0C0} & 2 & 27٫5 & 20٫8& 17٫7 \\
		\multirow{-3}{*}{\cellcolor[HTML]{C0C0C0}نقطهٔ شارژی دوم} & 3 & 15٫6& 12٫3 & 11,4 \\ \hline
	\end{tabular}%%
	}
}
\newcommand{\mytabl}{% Just for this example
	\resizebox{0.45\textwidth}{!}{%
		\begin{tabular}{|
				>{\columncolor[HTML]{C0C0C0}}c |
				>{\columncolor[HTML]{EFEFEF}}c |ccc|}
			\hline
			\multicolumn{2}{|c|}{\cellcolor[HTML]{C0C0C0}} & \multicolumn{3}{c|}{\cellcolor[HTML]{C0C0C0}نقطهٔ شارژی اول} \\ \cline{3-5} 
			\multicolumn{2}{|c|}{\multirow{-2}{*}{\cellcolor[HTML]{C0C0C0}زمان سرویس‌دهی در نقطهٔ شارژی اول}} & \cellcolor[HTML]{EFEFEF}1 & \cellcolor[HTML]{EFEFEF}2 & \cellcolor[HTML]{EFEFEF}3 \\ \hline
			\cellcolor[HTML]{C0C0C0} & 1 & 43٫4 & 35٫8 & 34٫6 \\
			\cellcolor[HTML]{C0C0C0} & 2 & 43٫4& 35٫9 & 34٫4 \\
			\multirow{-3}{*}{\cellcolor[HTML]{C0C0C0}نقطهٔ شارژی دوم} 
			& 3 & 43٫35& 35٫9 & 34٫5\\ \hline
		\end{tabular}%%%
	}
}
\newcommand{\mytable}{% Just for this example
	\resizebox{0.45\textwidth}{!}{%
			\begin{tabular}{|
					>{\columncolor[HTML]{C0C0C0}}c |
					>{\columncolor[HTML]{EFEFEF}}c |ccc|}
				\hline
				\multicolumn{2}{|c|}{\cellcolor[HTML]{C0C0C0}} & \multicolumn{3}{c|}{\cellcolor[HTML]{C0C0C0}نقطهٔ شارژی اول} \\ \cline{3-5} 
				\multicolumn{2}{|c|}{\multirow{-2}{*}{\cellcolor[HTML]{C0C0C0}زمان سرویس‌دهی در نقطهٔ شارژی دوم}} & \cellcolor[HTML]{EFEFEF}1 & \cellcolor[HTML]{EFEFEF}2 & \cellcolor[HTML]{EFEFEF}3 \\ \hline
				\cellcolor[HTML]{C0C0C0} & 1 & 10٫7 & 10٫5 & 10٫3\\
				\cellcolor[HTML]{C0C0C0} & 2 & 10٫1& 10٫1& 10٫0 \\
				\multirow{-3}{*}{\cellcolor[HTML]{C0C0C0}نقطهٔ شارژی دوم} 
				& 3 & 10٫0 & 10٫0  & 10٫0 \\ \hline
		\end{tabular}%%
	}
}




\makeatletter
% تعریف قلم فارسی و انگلیسی و مکان قلم‌ها
\if@irfonts
\settextfont[Path={./font/}, BoldFont={IRLotusICEE_Bold.ttf}, BoldItalicFont={IRLotusICEE_BoldIranic.ttf}, ItalicFont={IRLotusICEE_Iranic.ttf},Scale=1.2]{IRLotusICEE.ttf}
% LiberationSerif or FreeSerif as free equivalents of Times New Roman
\setlatintextfont[Path={./font/}, BoldFont={LiberationSerif-Bold.ttf}, BoldItalicFont={LiberationSerif-BoldItalic.ttf}, ItalicFont={LiberationSerif-Italic.ttf},Scale=1]{LiberationSerif-Regular.ttf}
% چنانچه می‌خواهید اعداد در فرمول‌ها، انگلیسی باشد، خط زیر را غیرفعال کنید
% و گزینهٔ displaymathdigits=persian را از خط ۱۰۹ حذف کنید.
\setdigitfont[Path={./font/}, Scale=1.2]{IRLotusICEE.ttf}
% تعریف قلم‌های فارسی و انگلیسی اضافی برای استفاده در بعضی از قسمت‌های متن
\setiranicfont[Path={./font/}, Scale=1.3]{IRLotusICEE_Iranic.ttf}				% ایرانیک، خوابیده به چپ
\setmathsfdigitfont[Path={./font/}]{IRTitr.ttf}
\defpersianfont\titlefont[Path={./font/}, Scale=1]{IRTitr.ttf}
% برای تعریف یک قلم خاص عنوان لاتین، خط بعد را فعال و ویرایش کنید و خط بعد از آن را غیرفعال کنید.
% \deflatinfont\latintitlefont[Scale=1]{LiberationSerif}
\font\latintitlefont=cmssbx10 scaled 2300 %cmssbx10 scaled 2300
\else
\settextfont{XB Niloofar}
\setlatintextfont{Junicode}
% چنانچه می‌خواهید اعداد در فرمول‌ها، انگلیسی باشد، خط زیر را غیرفعال کنید
% و گزینهٔ displaymathdigits=persian را از خط ۱۰۹ حذف کنید.
\setdigitfont{XB Niloofar}
% تعریف قلم‌های فارسی و انگلیسی اضافی برای استفاده در بعضی از قسمت‌های متن
% \setmathsfdigitfont{XB Titre}
\defpersianfont\titlefont{XB Titre}
\deflatinfont\latintitlefont[Scale=1.1]{Junicode}
\fi
\makeatother

% برای استفاده از قلم نستعلیق خط بعد را فعال کنید.
% \defpersianfont\nastaliq[Scale=1.2]{IranNastaliq}


%%%%%%%%%%%%%%%%%%%%%%%%%%
% راستچین شدن todonotes
\presetkeys{todonotes}{align=right,textdirection=righttoleft}{}
\makeatletter
\providecommand\@dotsep{5}
\def\listtodoname{فهرست کارهای باقیمانده}
\def\listoftodos{\noindent{\Large\vspace{10mm}\textbf{\listtodoname}}\@starttoc{tdo}}
\renewcommand{\@todonotes@MissingFigureText}{شکل}
\renewcommand{\@todonotes@MissingFigureUp}{شکل}
\renewcommand{\@todonotes@MissingFigureDown}{جاافتاده}
\makeatother
% دستوری برای حذف کلمه «چکیده»
\renewcommand{\abstractname}{}
% دستوری برای حذف کلمه «abstract»
%\renewcommand{\latinabstract}{}
% دستوری برای تغییر نام کلمه «اثبات» به «برهان»
\renewcommand\proofname{\textbf{برهان}}
% دستوری برای تغییر نام کلمه «کتاب‌نامه» به «مراجع»
\renewcommand{\bibname}{مراجع}
% دستوری برای تعریف واژه‌نامه انگلیسی به فارسی

% تعریف دستور جدید «\پ» برای خلاصه‌نویسی جهت نوشتن عبارت «پروژه/پایان‌نامه/رساله»
\newcommand{\پ}{پروژه/پایان‌نامه/رساله }

%\newcommand\BackSlash{\char`\\}

%%%%%%%%%%%%%%%%%%%%%%%%%%
% \SepMark{-}

% تعریف و نحوه ظاهر شدن عنوان قضیه‌ها، تعریف‌ها، مثال‌ها و ...
\theoremstyle{definition}
\newtheorem{definition}{تعریف}[section]
\theoremstyle{theorem}
\newtheorem{theorem}[definition]{قضیه}
\newtheorem{lemma}[definition]{لم}
\newtheorem{proposition}[definition]{گزاره}
\newtheorem{corollary}[definition]{نتیجه}
\newtheorem{remark}[definition]{ملاحظه}
\theoremstyle{definition}
\newtheorem{example}[definition]{مثال}

%\renewcommand{\theequation}{\thechapter-\arabic{equation}}
%\def\bibname{مراجع}
\numberwithin{algorithm}{chapter}
\def\listalgorithmname{فهرست الگوریتم‌ها}
\def\listfigurename{فهرست تصاویر}
\def\listtablename{فهرست جداول}

%%%%%%%%%%%%%%%%%%%%%%%%%%%%
%%% دستورهایی برای سفارشی کردن سربرگ صفحات:
%\newcommand{\SetHeader}[1]{
% دستور زیر معادل با گزینه twoside است.
%\csname@twosidetrue\endcsname
\pagestyle{fancy}
%% دستورات زیر سبک صفحات fancy را تغییر می‌دهد:
% O=Odd, E=Even, L=Left, R=Right
% در صورت oneside بودن، عنوان فصل، سمت چپ ظاهر می‌شود.
%\fancyhead{}
%\fancyhead[OL]{\small\leftmark}
%\fancyhead[ER]{\small\leftmark}
%\fancyhead[OR]{\footnotesize\rightmark}
%\fancyhead[EL]{\footnotesize\rightmark}

\fancyhead{}
\fancyhead[OR]{\leftmark}
\fancyhead[EL]{\rl{نگارش پایان‌نامه و رساله به کمک \lr{\XePersian}}}
\fancyhead[OL]{\thepage}
\fancyhead[ER]{\thepage}
\renewcommand{\headrulewidth}{0.75pt}
\cfoot{}



\renewcommand{\headrulewidth}{0.75pt}



% شکل‌دهی شماره و عنوان فصل در سربرگ
\renewcommand{\chaptermark}[1]{\markboth{فصل~\thechapter:\ #1}{}}
\makeatletter
\renewcommand{\rightmark}[1]{\@title}
\makeatother
%}
%%%%%%%%%%%%%%%%%%%%%%%%%%%%
%\def\MATtextbaseline{1.5}
%\renewcommand{\baselinestretch}{\MATtextbaseline}
\doublespacing
%%%%%%%%%%%%%%%%%%%%%%%%%%%%%
% دستوراتی برای اضافه کردن کلمه «فصل» در فهرست مطالب

\newlength\mylenprt
\newlength\mylenchp
\newlength\mylenapp

\renewcommand\cftpartpresnum{\partname~}
\renewcommand\cftchappresnum{\chaptername~}
\renewcommand\cftchapaftersnum{:}

\settowidth\mylenprt{\cftpartfont\cftpartpresnum\cftpartaftersnum}
\settowidth\mylenchp{\cftchapfont\cftchappresnum\cftchapaftersnum}
\settowidth\mylenapp{\cftchapfont\appendixname~\cftchapaftersnum}
\addtolength\mylenprt{\cftpartnumwidth}
\addtolength\mylenchp{\cftchapnumwidth}
\addtolength\mylenapp{\cftchapnumwidth}

\setlength\cftpartnumwidth{\mylenprt}
\setlength\cftchapnumwidth{\mylenchp}	

\makeatletter
{\def\thebibliography#1{\chapter*{\refname\@mkboth
   {\uppercase{\refname}}{\uppercase{\refname}}}\list
   {[\arabic{enumi}]}{\settowidth\labelwidth{[#1]}
   \rightmargin\labelwidth
   \advance\rightmargin\labelsep
   \advance\rightmargin\bibindent
   \itemindent -\bibindent

   \listparindent \itemindent
   \parsep \z@
   \usecounter{enumi}}
   \def\newblock{}
   \sloppy
   \sfcode`\.=1000\relax}}
   
%اگر مایلید در شماره گذاری حرفی و ابجد به جای آ از الف استفاده شود دستورات زیر را فعال کنید.   
%\def\@Abjad#1{%
%  \ifcase#1\or الف\or ب\or ج\or د%
%           \or هـ\or و\or ز\or ح\or ط%
%           \or ی\or ک\or ل\or م\or ن%
%           \or س\or ع\or ف\or ص%
%           \or ق\or ر\or ش\or ت\or ث%
%            \or خ\or ذ\or ض\or ظ\or غ%
%            \else\@ctrerr\fi}
%
% \def\abj@num@i#1{%
%   \ifcase#1\or الف\or ب\or ج\or د%
%            \or هـ‍\or و\or ز\or ح\or ط\fi

%   \ifnum#1=\z@\abjad@zero\fi}   
%  
%   \def\@harfi#1{\ifcase#1\or الف\or ب\or پ\or ت\or ث\or

% ج\or چ\or ح\or خ\or د\or ذ\or ر\or ز\or ژ\or س\or ش\or ص\or ض\or ط\or ظ\or ع\or غ\or

% ف\or ق\or ک\or گ\or ل\or م\or ن\or و\or ه\or ی\else\@ctrerr\fi}

%

%%% امکان درج کد در سند
% در این قسمت رنگ، قلم و قالب‌بندی قسمت‌های مختلف یک کد تعیین می‌شود. 
\lstdefinestyle{myStyle}{
	basicstyle=\ttfamily, % whole listing /w verbatim font
	keywordstyle=\color{blue}\bfseries, % bold black keywords
	identifierstyle=, % nothing happens
	commentstyle=\color{LimeGreen}, % green comments
	stringstyle=\ttfamily\color{red}, % red typewriter font for strings
	showstringspaces=false % no special string spaces
	breaklines=true,
	breakatwhitespace=false,
	numbers=right, % line number formats
	numberstyle=\footnotesize\lr,
	numbersep=-10pt,
	frame=single,
	captionpos=b,
	captiondirection=RTL
}
\lstset{style=myStyle} % command to set default style
\def\lstlistingname{\rl{برنامهٔ}}
\def\lstlistlistingname{\rl{فهرست برنامه‌ها}}
\pagestyle{empty}
\usepackage{afterpage}
\newcommand\blankpage{%
	\null
	\thispagestyle{empty}%
	\addtocounter{page}{-1}%
	\newpage}

% for numbering subsubsections
\setcounter{secnumdepth}{3}
%to include subsubsections in the table of contents
\setcounter{tocdepth}{3}
